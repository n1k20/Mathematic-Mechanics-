\documentclass[12pt, a4paper]{article}
\usepackage[utf8]{inputenc}
\usepackage[T2A]{fontenc}
\usepackage[english, russian]{babel}
\usepackage{amsmath, amssymb, amsthm}
\usepackage{mathtools}               
\usepackage{bm}                    
\usepackage{mathrsfs}
\usepackage{graphicx}
\usepackage{float}
\usepackage{array, booktabs}
\usepackage{cmap} %улучшает копирование и поиск русского текста в PDF-документе;
\usepackage[T2A]{fontenc} % используется для корректного отображения кириллицы;
\usepackage[utf8]{inputenc} % определяет юникодовую кодировку документа с кодом;
\usepackage[english,russian]{babel} % улучшает расстановку переносов и сокращений английских и русских слов;
\usepackage{amsmath,amsfonts,amssymb,amsthm,mathtools} % набор пакетов для расширения функционала при работе с формулами;
\usepackage[left=3cm,right=1cm,top=2cm,bottom=2cm]{geometry} % задаёт поля страницы (в данном случае слева — 3 см, справа — 1 см, сверху — 2 см, снизу — 2 см);
\usepackage{graphicx} % расширяет функционал при работе с рисунками;
\usepackage[colorlinks,linkcolor=blue]{hyperref} % цвет ссылок в тексте становится синим;
\usepackage{physics} % повышает удобство записи физических формул благодаря кратким и понятным командам;
\usepackage{esvect} % даёт возможность рисовать красивый значок вектора;
\usepackage{icomma} % делает дополнительный пробел после запятой в математическом режиме в PDF-файле, если после неё есть пробел в файле с кодом;
\usepackage{indentfirst} % делает красную строку в первом абзаце раздела;
\usepackage[labelsep=period,justification=centering]{caption} % форматирует подписи к рисункам и таблицам, согласно ГОСТ.
\usepackage{titlesec}

\titleformat{\section}{\filcenter\normalfont\Large\bfseries}{\thesection.}{0.2em}{}
\titleformat{\subsection}{\filcenter\normalfont\large\bfseries}{\thesubsection.}{1em}{}
\titleformat{\subsubsection}{\filcenter\normalfont\normalsize\bfseries}{\thesubsubsection.}{1em}{}

\theoremstyle{plain}
\newtheorem{theorem}{Теорема}[section]
\newtheorem{lemma}{Лемма}[section]
\newtheorem{corollary}{Следствие}[section]

\theoremstyle{definition}
\newtheorem{definition}{Определение}[section]
\newtheorem{example}{Пример}[section]

\theoremstyle{remark}
\newtheorem{remark}{Замечание}[section]

\title{Математический Анализ}
\author{Боргояков Николай Алексеевич}


\begin{document}
    Прокуди Дмитрий Алексеевич

    телефон: 89529406790

    e-mail: dmitriy179354@mail.ru 

    \section{Глава Множества и Отображения}
    \subsection{Множества}
    \subsubsection{п. Понятие множества}
    Множестав -- это совокупность различимых объектов произвольной природы,
    рассматриваемая как единое целое. Сами объекты называются элементами множества.

    Для обозначения множеств будем использовать прописные буквы латинского алфавита $(A, B, C,\ldots),$
    a для элементов строчные $(a, b, c, \ldots)$.

    Тот факт, что некоторый объект $x$ является элементом множества $A$, записывают 
    $x \in A $ Если же $x$ не является элементом $A$, записывают $x \notin A$

    B в качестве синонимов синонимов термина "множество" будем использовать термины 
    "класс" , "семейство" , "система" , "набор" и др.


    \textbf{Способы задания множества:}
    \begin{enumerate}
        \item Если множества $A$ состоят из конечного числа элементов,
        то можно просто все эти элементы перечислить, записав их в фигурных 
        скобках через запятую.
        Например, если $A$ есть множество букв, составляющих слово "математика", то
        $A = \{\text{м}, \text{а}, \text{т}, \text{е}, \text{и}, \text{к}\}$
        
        \item Пусть $P$ -- какое-либо свойство, и запись $P(x)$, означает, что объект
        $x$ обладает свойством $P$. Тот факт, что $A$ есть множество объектов,
        обладающих свойством $P$, записывают $A = \{x | P(x)\}$
        $|$ -- это обозначение слов, \glqq которые\grqq  или  \glqq такие что\grqq 
    \end{enumerate}

    Если мы дополнительно потребуем, чтобы эти объекты выбирались из некоторого 
    другого множества $B$, то запишем $A = \{ x \in B | P(x)\}$

    Читается это так: множество $A$ состоит из тех (и только тех) элементов 
    множества $B$, которые обладают свойством $P$.

    Пример. Пусть $A$ -- множество букв русского алфавита. Тогда множество 
    гласных букв $B = {x \in A | x - \text{гласная}}$. Запишем, что здесь 
    можно воспользоваться способом 1): $B = \{\text{а}, \text{е}, \text{ё}, 
    \text{и}, \text{о}, \text{у}, \text{ы}, \text{э}, \text{ю}, \text{я}\}$.
    
    Множества $A$ и $B$ равны (обозначение: $A = B$), если они состоят из 
    одних и тех же элементов. Выражение $A \neg B$ означает, что множества 
    $A$ и $B$ не равны, т.e. не все элементы одного множества являются 
    элементами другого .

    Равенство есть отношение эквивалентности между множествами, т.к. оно 
    обладает следующими свойствами:
    \begin{enumerate}
        \item рефлексивностью $(A = A)$
        \item симметричностью (если $A = B$, то $B = A$)
        \item транзитивностью (если $A = B$ и $B = C$, то $A = C$)
    \end{enumerate}

    Если каждый элемент множества $A$ является элементом множества 
    $B$, то говорят, что $A$ является подмножеством множества $B$ 
    (или $A$ содержится в $B$). Обозначим: $A \subset B$

    \textbf{Свойства:}
    \begin{enumerate}
        \item $A \subset A$;
        \item если $A \subset B$ и $B \subset A$, то $A = B$;
        \item если $A \subset B$ и $B \subset C$, то $A \subset C$
    \end{enumerate}

    Удобно ввести множество, совсем не имеющие элементов и 
    называемое пустым множеством . Обозначение: $\emptyset$
    
    Пустое множество является подмножеством любого множества.

    \subsubsection{п. Логическая символика}

    Утверждение -- это высказывание, которое может быть 
    либо истинным, либо ложным. Пусть $A$ и $B$ утверждения, Тогда

    \begin{enumerate}
        \item Утверждение $\lnot A$ (не $A$) истинно тогда и 
        только тогда, когда $A$ ложно;
        \item Утверждение $A \land B$ ($A$ и $B$) истинно 
        тогда и только тогда, когда $A$ и $B$ истинны;
        \item Утверждение $A \lor B$ ($A$ или $B$) истинно 
        тогда и только тогда, когда истинно хотя бы одно из утверждений
        $A$ и $B$ истинно;
        \item Утверждение $A \Rightarrow B$ (из $A$ следует $B$) означает,
        что если $A$ истинно, то и $B$ -- истинно;
        \item Утверждение $A \Leftrightarrow B$ ($A$ равносильно $B$) означает,
        что из истинности $A$ следует истинности $B$ и из истинности $B$
        следует истинности $A$;
    \end{enumerate}

    В формулировках утверждений часто используется следующее: \glqq если $A$, то $B$ \grqq,
    \glqq для того, чтобы $A$, необходимо, что $B$ \grqq, \glqq для того, чтобы $B$,
    достаточно, чтобы $A$ \grqq, что означает $A \Rightarrow B$



\end{document}